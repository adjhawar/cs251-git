\documentclass[a4paper]{article}
\usepackage{amsmath}
\usepackage{amssymb}
\usepackage{geometry}
\geometry{left=2.5cm,right=2.5cm,top=2.5cm,bottom=2.5cm}

\title{Oh My God! }

\author{Vaibhav Sarvesh Pandey, 13761\\

	Rahul Kumar,150548\\
	Aditya Jhawar, 150050}
% 	\ttfamily{terwangjindong@ict.ac.cn}}


\begin{document}
	\maketitle
	

\section{OMG: Oh My God! }
\subsection{Plot}
Kanji Lalji Mehta (Paresh Rawal), a middle-class atheist Hindu owns a shop of Hindu idols and antiques in Mumbai. He is cursed by Siddheshwar Maharaj when he stopped his son from playing dahi handi. A low-intensity earthquake hits the city, and Kanji's shop is the only shop that is destroyed. Qureshi, a disabled man, helps him file the case as Kanji decides to fight on his own. Legal notices are sent to the insurance company as well as to religious priests, Siddheshwar Maharaj (Govind Namdeo), Gopi Maiyya (Poonam Jhawer) and their group's founder, Leeladhar Swamy (Mithun Chakraborty) summoning them to the court as representatives of God on earth.
Next day at the insurance office, Kanji and his neighbour-assistant Mahadev learn that the disaster claim does not cover any damage caused by natural calamities classified under "Act of God".   Running out of options, Kanji decides to file a lawsuit against God but fails to find a lawyer for such a lawsuit. Finally, he meets Hanif Qureshi (Om Puri) and his daughter (Puja Gupta), a poor Muslim lawyer family.
	\subsection{Music }
The sgs are few in number and are in just the right places. they take the story forward without adding to unneccesary drama. 'Mere Nishaan', sung by Kailash Kher and Meet Brothers-Anjjan is worth listening to. Shreya Ghoshal and Mika have sung 'Go Go Govinda' to perfection. Sonakshi and Prabhu Deva look good as they dance to its tune. 
	        
	  \subsection{cast}
	        
Akshay Kumar as Krishna Vasudev Yadav / Lord Krishna
Paresh Rawal as Kanji Lalji Mehta
Mithun Chakraborty as Leeladhar Swamy
Govind Namdeo as Siddheshwar Maharaj
Poonam Jhawer as Gopi Maiyya
Om Puri as Advocate Hanif Qureshi
Mahesh Manjrekar as Advocate Sardesai
Puja Gupta as Hanif's Daughter                  
	  

	   
	
	   
	
	
	  \section{Lagaan }
	  'Lagaan' which literally means Tax is a story of ordinary people set in a small village 'Champaner' in Central India during the British rule in 1893. The tale narrates the determination of the villagers to fight against injustice and opposition.
	
	\subsection{PLOT}
Kanji Lalji Mehta (Paresh Rawal), a middle-class atheist Hindu owns a shop of Hindu idols and antiques in Mumbai. He is cursed by Siddheshwar Maharaj when he stopped his son from playing dahi handi. A low-intensity earthquake hits the city, and Kanji's shop is the only shop that is destroyed.	
	
	
	
	  \section{Lagaan }
	  Bhuvan (Aamir Khan) a young farmer takes up the impossible task to save his village. He forms a team of 11 with his fellowmen. Elizabeth (Rachel Shelly), Captain Russell's sister who has a soft corner for Bhuvan gives her full support to the villagers cause despite her brother's opposition. Gauri (Gracy Singh), a village girl loves Bhuvan and has complete faith in him.

The music in this film composed by A.R.Rehman is radically different from other films, it transports us back to the days of the British Raj and gives a right feel of the time. The lyricist and singers have also contributed to the earthy feel and rural touch of the music. Aamir Khan as Bhuvan gives a convincing performance as well. Gracy Singh, has merged with the character of Gauri as a lively and charming village belle. British actors : Paul Blackthorne (Russel) and Rachel Shelley (Elizabeth) both stage artists from Britain and other Indian actors Yashpal Sharma, Raghubir Yadav, Rajesh Vivek, Aditya Lakhia, Amin Hajee have all put in their best.

'Lagaan' shot in and around the Kutch of Bhuj in Gujarat (where the disastrous earth quake took place in 2001) is Aamir Khan's first production. It is commendable that he has made extensive research, even filmed period pieces in the U.K to give a feel of authenticity to the film. Around 40 British actors have played the role of Englishmen and officers.

The cricket sequences have also been taken very well and Ashutosh Gowariker gives a very emotional end to the movie. In contrast to many bollywood movies this is worth a watch, even though it lasts for about 3 hours and 40 minutes.
	  
	  
	  


	    	        
	        
	            
	        
	     
	    
	

	
	
\end{document}
