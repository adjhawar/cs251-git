\documentclass[a4paper]{article}
\usepackage{amsmath}
\usepackage{amssymb}
\usepackage{geometry}
\geometry{left=2.5cm,right=2.5cm,top=2.5cm,bottom=2.5cm}

\title{Oh My God! }
\author{Neeraj Kumar, 13427\\
	Vaibhav Sarvesh Pandey, 13761\\
Rahul Kumar,150548\\
\title{Movie1 }
\author{Vaibhav Sarvesh Pandey, 13761\\

	Aditya Jhawar, 150040}
% 	\ttfamily{wangjindong@ict.ac.cn}}

\begin{document}
	\maketitle
	
	\section{OMG: Oh My God! }
	
	\subsection{PLOT}
Kanji Lalji Mehta (Paresh Rawal), a middle-class atheist Hindu owns a shop of Hindu idols and antiques in Mumbai. He is cursed by Siddheshwar Maharaj when he stopped his son from playing dahi handi. A low-intensity earthquake hits the city, and Kanji's shop is the only shop that is destroyed.	
	
	
	
	  \section{Lagaan }
	  'Lagaan' which literally means Tax is a story of ordinary people set in a small village 'Champaner' in Central India during the British rule. The tale narrates the determination of the villagers to fight against injustice and opposition.
	  
	  There is no rain in the village for about a year. The villagers who depend on rain for farming are unable to pay the lagaan to the ruler Captain Russel. On top of that they are asked to pay double lagaan of what they usually pay. Captain Russell gives an alternative to the villagers that if they defeat the British in a game of cricket, they will be exempted from tax for three consecutive years but if they lose they will have to pay three times more lagaan. It becomes a situation of life or death for the villagers.
	  
	  


	    	        
	        
	            
	        
	        
	        
	    
	    
	

	
	
\end{document}
