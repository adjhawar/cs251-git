\documentclass[a4paper]{article}
\usepackage{amsmath}
\usepackage{amssymb}
\usepackage{geometry}
\geometry{left=2.5cm,right=2.5cm,top=2.5cm,bottom=2.5cm}

\title{Oh My God! and Lagaan}

\author{Vaibhav Sarvesh Pandey, 13761\\
	Rahul Kumar,150548\\
	Aditya Jhawar, 150050}
% 	\ttfamily{terwangjindong@ict.ac.cn}}


\begin{document}
	\maketitle
	

\section{OMG: Oh My God! }
\subsection{Plot}

Kanji Lalji Mehta (Paresh Rawal), a middle-class atheist Hindu owns a shop of Hindu idols and antiques in Mumbai. He is cursed by Siddheshwar Maharaj when he stopped his son from playing dahi handi. A low-intensity earthquake hits the city, and Kanji's shop is the only shop that is destroyed. Qureshi, a disabled man, helps him file the case as Kanji decides to fight on his own. Legal notices are sent to the insurance company as well as to religious priests, Siddheshwar Maharaj (Govind Namdeo), Gopi Maiyya (Poonam Jhawer) and their group's founder, Leeladhar Swamy (Mithun Chakraborty) summoning them to the court as representatives of God on earth.
Next day at the insurance office, Kanji and his neighbour-assistant Mahadev learn that the disaster claim does not cover any damage caused by natural calamities classified under "Act of God".   Running out of options, Kanji decides to file a lawsuit against God but fails to find a lawyer for such a lawsuit. Finally, he meets Hanif Qureshi (Om Puri) and his daughter (Puja Gupta), a poor Muslim lawyer family.
	
\subsection{Music }
The songs are few in number and are in just the right places. they take the story forward without adding to unneccesary drama. 'Mere Nishaan', sung by Kailash Kher and Meet Brothers-Anjjan is worth listening to. Shreya Ghoshal and Mika have sung 'Go Go Govinda' to perfection. Sonakshi and Prabhu Deva look good as they dance to its tune.   
  
	  \subsection{Cast}
	        
Akshay Kumar as Krishna Vasudev Yadav / Lord Krishna
Paresh Rawal as Kanji Lalji Mehta
Mithun Chakraborty as Leeladhar Swamy
Govind Namdeo as Siddheshwar Maharaj
Poonam Jhawer as Gopi Maiyya
Om Puri as Advocate Hanif Qureshi
Mahesh Manjrekar as Advocate Sardesai
Puja Gupta as Hanif's Daughter                  
	  
Kanji Lalji Mehta (Paresh Rawal), a middle-class atheist Hindu owns a shop of Hindu idols and antiques in Mumbai. He is cursed by Siddheshwar Maharaj when he stopped his son from playing dahi handi. A low-intensity earthquake hits the city, and Kanji's shop is the only shop that is destroyed. Next day at the insurance office, Kanji and his neighbour-assistant Mahadev learn that the disaster claim does not cover any damage caused by natural calamities classified under "Act of God".   Running out of options, Kanji decides to file a lawsuit against God but fails to find a lawyer for such a lawsuit.\\
	Finally, he meets Hanif Qureshi (Om Puri) and his daughter (Puja Gupta), a poor Muslim lawyer family.Qureshi, a disabled man, helps him file the case as Kanji decides to fight on his own. Legal notices are sent to the insurance company as well as to religious priests, Siddheshwar Maharaj (Govind Namdeo), Gopi Maiyya (Poonam Jhawar) and their group's founder, Leeladhar Swamy (Mithun Chakraborty) summoning them to the court as representatives of God on earth.	        
	                  



\subsection{Audience Response}
	   
	The film was critically well received.
Taran Adarsh of Bollywood Hungama gave the film 3.5 out of 5 stars and said "On the whole, OMG – OH MY GOD! is a thought-provoking adaptation of a massively successful play. A movie tackling a sensitive and an untouched subject matter, it will find its share of advocates and adversaries, but the social message the movie conveys comes across loud and clear and that's one of the prime reasons why OMG becomes a deserving watch". Sukanya Verma of Rediff rated it 4 out of 5 reviewing "A brave and absorbing blend of satire, fable and fantasy that brings our attention to the misuse and commercialisation of religion". Faisal Saif of Independent Bollywood rated it 4 out of 5 and said "Strongly Recommended. Fearless concept with some Fearless performances".   	        
	        
	  	
	
	
	  \section{Lagaan }
	  
	  
	 
	  \subsection {Cast}
	  \begin{itemize}
\item Aamir Khan as Bhuvan
\item Rachel Shelly as Elizabeth
\item Gracy Singh as Gauri
\item Pual Blackthorne as Russel
\end{itemize}
	  \subsection{Plot }
    
	  'Lagaan' which literally means Tax is a story of ordinary people set in a small village 'Champaner' in Central India during the British rule in 1893. The tale narrates the determination of the villagers to fight against injustice and opposition.
	  \\
	  There is no rain in the village for a year. The villagers who depend on rain for farming are unable to pay the lagaan to the ruler Captain Russel. On top of that they are asked to pay double of what they usually pay. Captain Russell gives an alternative to the villagers that if they defeat the British in a game of cricket, they will be exempted from tax for three consecutive years but if they loose they will have to pay three times more lagaan. It becomes a life or death situation for the villagers.
\\	  
	  Bhuvan (Aamir Khan) a young farmer takes up the impossible task to save his village. He forms a team of 11 with his fellowmen. Elizabeth (Rachel Shelly), Captain Russell's sister who has a soft corner for Bhuvan gives her full support to the villagers cause despite her brother's opposition. Gauri (Gracy Singh), a village girl loves Bhuvan and has complete faith in him.


\subsection{Music}


The music in this film composed by A.R.Rehman is radically different from other films, it transports us back to the days of the British Raj and gives a right feel of the time. The lyricist and singers have also contributed to the earthy feel and rural touch of the music. Aamir Khan as Bhuvan gives a convincing performance as well. Gracy Singh, has merged with the character of Gauri as a lively and charming village belle. British actors : Paul Blackthorne (Russel) and Rachel Shelley (Elizabeth) both stage artists from Britain and other Indian actors Yashpal Sharma, Raghubir Yadav, Rajesh Vivek, Aditya Lakhia, Amin Hajee have all put in their best.


\subsection{Direction}
'Lagaan' shot in and around the Kutch of Bhuj in Gujarat (where the disastrous earth quake took place in 2001) is Aamir Khan's first production. It is commendable that he has made extensive research, even filmed period pieces in the U.K to give a feel of authenticity to the film. Around 40 British actors have played the role of Englishmen and officers.

"Lagaan" somehow succeeds in being suspenseful but at the same time it's frivolous and obvious. The final cricket match (which we can follow even if we don't understand the game) is in the time-honored tradition of all sports movies, and yet the underlying issues are serious. The director Ashutosh Gowariker is not shy about lingering on ancient forts and palaces, vast plains, and the architecture of the British Raj, so out of place and yet so serenely confident.

The cricket sequences have also been taken very well and Ashutosh Gowariker gives a very emotional end to the movie. In contrast to many bollywood movies this is worth a watch, even though it lasts for about 3 hours and 40 minutes.
	  

	  

	\subsection{Audience Responses}	            
	Lagaan was met with high critical acclaim. The film currently scores a 95 percent "Certified Fresh" approval rating on review aggregate site Rotten Tomatoes, based on 59 reviews, with an average rating of 7.9/10. The site's critical consensus is, "Lagaan is lavish, rousing entertainment in the old-fashioned tradition of Hollywood musicals." Derek Elley of Variety suggested that it "could be the trigger for Bollywood's long-awaited crossover to non-ethnic markets". Somni Sengupta of The New York Times, described it as "a carnivalesque genre packed with romance, swordplay and improbable song-and-dance routines. Roger Ebert gave three and half out of four stars and said, "Lagaan is an enormously entertaining movie, like nothing we've ever seen before, and yet completely familiar... At the same time, it's a memory of the films we all grew up on, with clearly defined villains and heroes, a romantic triangle, and even a comic character who saves the day. Lagaan is a well-crafted, hugely entertaining epic that has the spice of a foreign culture." Peter Bradshaw of The Guardian described the film as "a lavish epic, a gorgeous love story, and a rollicking adventure yarn. Larger than life and outrageously enjoyable, it's got a dash of spaghetti western, a hint of Kurosawa, with a bracing shot of Kipling. Kuljinder Singh of the BBC stated that "Lagaan is anything but standard Bollywood fodder, and is the first must-see of the Indian summer. A movie that will have you laughing and crying, but leaving with a smile." Kevin Thomas of the Los Angeles Times argued that the film is "an affectionate homage to a popular genre that raises it to the level of an art film with fully drawn characters, a serious underlying theme, and a sophisticated style and point of view .       
	     
	    
	

	  


	    	   
	

	
	
\end{document}
